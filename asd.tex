% This is a simple sample document.  For more complicated documents take a look in the exercise tab. Note that everything that comes after a % symbol is treated as comment and ignored when the code is compiled.

\documentclass{article} % \documentclass{} is the first command in any LaTeX code.  It is used to define what kind of document you are creating such as an article or a book, and begins the document preamble

\usepackage{amsmath} % \usepackage is a command that allows you to add functionality to your LaTeX code

\title{Simple Sample} % Sets article title
\author{My Name} % Sets authors name
\date{\today} % Sets date for date compiled

% The preamble ends with the command \begin{document}
\begin{document} % All begin commands must be paired with an end command somewhere
\section{Bevezetés}

		A szakdolgozatom egy weboldal, amely alkalmas többféle pénzügyi adat elemzéséhez, ehhez különféle befektetési alapokat használok, továbbá egy grafikonrajzolót. Ennek az oldalnak a szervezeti felépítését, elkészítését és az alkalmazás működését mutatom be.
A pénzügyi adatok, vagy tőzsdei árfolyamok megismerése lehet nagyon egyszerű, de gyakran nehézséget jelent azoknak, akik előszőr próbálkoznak meg vele a hétköznapokban. Ezért a dolgozatom olyan lehetőségeket mutat be, amelyekkel átláthatóbban lehet megérteni és kezelni őket.

	Ezek a lehetőségeket kétféle alternatívára osztottam szét. Az egyik opció a részletes leírása és ismertetése a befektetési alapoknak, kiegészítve egy táblázattal, amely ábrázolja a hozam-kockázat profilt, továbbá egy alap által elért éves nettó hozam sáv. A másik lehetőségként külön oldalon található grafikonrajzoló használható. Lehetőség van kezdési és végpontot megadni, így különböző intervallumokat kiválasztva tudjuk vizsgálni az adott befektetési alapot. A dolgozat az utóbbira fektet nagyobb hangsúlyt több okból kifolyólag is. Ezt elsősorban az indokolja, hogy az így nyert adatokat rugalmasabban tudja kezelni a felhasználó, mivel egyszerre több kiválasztási lehetőség áll rendelkezésre, amivel részletesebb adathalmazt nyerhető ki.

	A megjelenítéshez és az algoritmusok fejlesztéséhez több programozási nyelvet is került választásra, többek között HTML5 képezi a weboldal vázát, CSS a weboldal megjelenítését képezi, Node.JS amellyel a weboldal szervere működik és végezetül JavaScript nyelven válik elérhetővé a grafikonrajzoló.

	Több grafikonrajzoló és eszközkezelő weboldal elérhető a különböző befektetési vállalatoknak az interneten.  Ebből kifolyólag felmerülhet a kérdés, hogy akkor miért volt szükség még egy weboldal elkészítésére? Többek között erre a kérdésre is választ kaphatunk a szakdolgozat elolvasása után.
    For a new paragraph I can leave a blank space in my code. 

\end{document} % This is the end of the document