%A szükséges csomagok
%méretek: margó, lap, fej~ és lábléc stb. NE MÓDOSÍTSUK!
\usepackage[paperwidth=210mm,paperheight=297mm,headheight=15.2pt,headsep=7.1mm,top=25mm,bottom=25mm,inner=30mm,outer=25mm,dvips]{geometry}
%fejléc kezelõ csomag
\usepackage{fancyhdr}
%ams matematikai makrócsomag
\usepackage{amsmath}
%ams matematikai szimbólumcsomag
\usepackage{amssymb}
%színkezelés
\usepackage{color}
%eps ábrák beszúrása
\usepackage{epsfig}
%grafikai csomag
\usepackage{graphicx}
%vezérlési szerkezetek
\usepackage{ifthen}
%array csomag a táblázatokhoz
\usepackage{array}
%saját tételszerû környezetekhez
\usepackage{ntheorem}
%multirow csomag: összevont sorokhoz
\usepackage{multirow}
%szines tablazatok
\usepackage{colortbl}
%ábrakészítés a Tex-kel
\usepackage[metapost]{mfpic}
%latin2 kódolás - a magyar ékezetes betûk miatt
\usepackage[utf8]{inputenc}
%a magyar elválasztáshoz szükséges
\usepackage[T1]{fontenc}
%elválasztást végzõ csomag
\usepackage[magyar]{babel}

% \newcommand{\myinclude}[1]{%
% 
% \ifthenelse{\not\(\isodd{\thepage}\)}{\thispagestyle{empty}\hbox{}\newpage}%
% 
% \include{#1}}

%myheadings beállítása
\def\ps@myheadings{%
     \def\@oddfoot{\hfil\thepage}
     \def\@oddhead{}
      }

%fancyhdr beállítása
\fancyhead[L]{}
\fancyhead[R]{}
\fancyfoot[C]{}
\fancyfoot[R]{\thepage}

%chapter átdefiniálása
\renewcommand\chapter{\if@openright\cleardoublepage\else\clearpage\fi
                    \thispagestyle{myheadings}%
                    \global\@topnum\z@
                    \@afterindentfalse
                    \secdef\@chapter\@schapter}

%Saját fejezet parancs: \Chapter
\newcommand{\Chapter}[1]{\chapter{#1}
\chead{}
}

%Saját fejezet parancs, rövidítést használhatunk a tartalomjegyzékben: \SChapter
\newcommand{\SChapter}[2]{\chapter[#2]{#1}}

%Alfejezet parancs: \Section
\newcommand{\Section}[1]{\section{#1}
\chead{\textsl{\thesection. #1}}
}

%Alfejezet parancs, rövidítést használhatunk a fejlécben  és a tartalomjegyzékben is: \SSection
\newcommand{\SSection}[3]{\section[#3]{#1}
\chead{\textsl{\thesection. #2}}}

%Al-alfejezet parancs: \SubSection
\newcommand{\SubSection}[1]{\subsection{#1}
}

%Al-alfejezet parancs, rövidítést használhatunk a tartalomjegyzékben: \SSubSection
\newcommand{\SSubSection}[2]{\subsection[#2]{#1}
}

%Saját referencia - nem teszi hozzá a névelõt, de a pontot igen!
\newcommand{\myref}[1]{\ref{#1}.}

%Saját magyar referencia - hozzáteszi a névelõt és a pontot!
\newcommand{\myaref}[1]{\aref{#1}.}

%táblázatok beállítása
\renewcommand{\arraystretch}{1.3}  %default: 1 Sorok közötti távolság nagyítása

\setlength{\arrayrulewidth}{0.6pt} %default: 0.6pt A vonalak vastagsága

%A túlcsordulási hibák jelzése a nyomtatási képben vastag, telített téglalappal
\overfullrule10pt

%pont után nem nagyobb a szóköz
\frenchspacing


%Új tételkörnyezetek
\newtheoremstyle{MYhun}%
{\item[\hskip\labelsep\theorem@headerfont ##2.\ ##1.]}%
{\item[\hskip\labelsep\theorem@headerfont ##2.\ ##1\ \textnormal{(##3).}]}

\newtheoremstyle{MYwithoutnumber}%
{\item[\hskip\labelsep\theorem@headerfont ##1.]}%
{\item[\hskip\labelsep\theorem@headerfont ##3.]}

\theoremstyle{MYhun}
\theorembodyfont{\it}
\newtheorem{theorem}{tétel}[chapter]
\theorembodyfont{\rm}\newtheorem{definition}[theorem]{definíció}
\theorembodyfont{\it}\newtheorem{lemma}[theorem]{lemma}
\theorembodyfont{\it}\newtheorem{corollary}[theorem]{következmény}
\theorembodyfont{\rm}\newtheorem{remark}[theorem]{megjegyzés}
\theorembodyfont{\rm}\newtheorem{example}[theorem]{példa}
\theoremstyle{MYwithoutnumber}
\theorembodyfont{\rm}\newtheorem{prooF}{Bizonyítás}

\def\QEDsign{\hfill$\Box$}
\newenvironment{proof}{\begin{prooF}}{\QEDsign\end{prooF}}